% $Log: abstract.tex,v $
% Revision 1.1  93/05/14  14:56:25  starflt
% Initial revision
% 
% Revision 1.1  90/05/04  10:41:01  lwvanels
% Initial revision
% 
%
%% The text of your abstract and nothing else (other than comments) goes here.
%% It will be single-spaced and the rest of the text that is supposed to go on
%% the abstract page will be generated by the abstractpage environment.  This
%% file should be \input (not \include 'd) from cover.tex.


% In this thesis, I designed and implemented a compiler which performs
% optimizations that reduce the number of low-level floating point operations
% necessary for a specific task; this involves the optimization of chains of
% floating point operations as well as the implementation of a ``fixed'' point
% data type that allows some floating point operations to simulated with integer
% arithmetic.  The source language of the compiler is a subset of C, and the
% destination language is assembly language for a micro-floating point CPU.  An
% instruction-level simulator of the CPU was written to allow testing of the
% code.  A series of test pieces of codes was compiled, both with and without
% optimization, to determine how effective these optimizations were.

In this project, I designed and implemented a feedback system which samples roll, pitch, and yaw coordinates, saves them in nonvolatile memory, and provides the saved coordinates as desired values when performing feedback during replication attempts; this involves the sampling of an integrated accelerometer, gyroscope, and magnetometer unit, processing the raw measurements through an implementation of Madgwick's gradient descent filter, and storing floating point numbers in a ferroelectric RAM for later retrieval.  Feedback is given to the user through four vibration motors placed 90 degrees from each other along the main axis of the bat.  The construction of the swing feedback system involved the physical modification of a plastic bat, the addition of reinforcing members from laser-cut acrylic, the addition of LEDs and pushbuttons for user interaction, and the addition of four vibration motors.  The code incorporates open-source code available through GitHub and is based on the Arduino programming environment for accelerated embedded systems development.
